\section{Johdokset}

\newcommand{\mc}[1]{\multicolumn{2}{@{}l}{#1}}

Jokaisesta nimisanasta voidaan johtaa laatusanat "-tOn ja "-llinen
"-johtimilla. Esimerkiksi

raha => rahaton, rahallinen


\begin{tabular}{@{}ll}
\mc{Laatusanoista joudetaan UUs-päätteinen ominaisuudennimi} \\
\\
vakoinen    & valkoisuus \\
valkoisempi & valkoisemmuus \\
valkoisin   & valkoisimmuus
\end{tabular}


\bigskip

Teonsanoista johdetaan nimisanoja\footnote{Jos toinen sarake (Sana) on
tyhjä, johtimella voidaan johtaa sana mistä tahansa sanasta, muuten
vain sellaisesta sanasta, joka taipuu samoin kuin toisessa sarakkeessa
oleva sana.}


\bigskip


\begin{tabular}{@{}llll}
Johdin & Sana & Esimerkki & Huomautuksia\\
       &      &           \\

jA    &       & punoja  & Tekijän nimi. \\
      &       & tekijä \\
      &       &        \\

jUUs  &       & punojuus \\
      &       & tekijyys \\
      &       & \\

mA    &       & punoma & Kolmas nimitapa. \\
      &       & tekemä \\
      &       &        \\

minen &       & punominen & Neljäs nimitapa. \\
      &       & tekeminen \\
      &       &        \\
      & punoa     & punonta \\
      & muistaa   & muistella, muisto \\
      & aavistaa  & aavistella, aavistus \\
      & juontaa   & juonto \\
      & nuolaista & nuolaisu \\
      & tekijä    & tekijätär, tekijättäryys
\end{tabular}
\bigskip

Muitakin johdoksia voi tehdä, mutta niitä ei ole dokumentoitu.

Tietenkin näin saadaan myös sellaisia sanoja, jotka ovat muodollisesti
oikein, mutta joita ei ole sanakirjassa.

