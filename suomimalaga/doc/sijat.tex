\section{Sijapäätteet}


\begin{tabular}{@{}|l|l|l|} \hline
Sija        & Yksikkö  & Monikko \\ \hline
nimentö     & valo     & valo+t \\
omanto      & valo+n   & valo+jen \\
kohdanto    &          & \\
osanto      & valo+a   & valo+ja \\
olento      & valo+na  & valo+ina \\
tulento     & valo+ksi & valo+iksi \\
sisäolento  & valo+ssa & valo+issa \\
sisäeronto  & valo+sta & valo+ista \\
sisätulento & valo+on  & valo+ihin \\
ulko-olento & valo+lla & valo+illa \\
ulkoeronto  & valo+lta & valo+ilta \\
ulkotulonto & valo+lle & valo+ille \\
vajanto     & valo+tta & valo+itta \\
seuranto    &          & valo+ine \\
keinonto    &          & valo+in \\ \hline
\end{tabular}
\bigskip

Monikon tunnus i tai j ja sijapääte on yhdistetty.

Sijapäätteillä on etu- ja takaääntiöinen muoto (kylvö+ön, valo+on).

Etuääntiöitä ovat ä, y ja ö, takaääntiöitä a, o ja u.
Ääntiöt e ja i eivät vaikuta sijapäätteen valintaan.

\bigskip


\begin{table}
Monikon nimennöllä sekä yksikön ja monikon omannolla, osannolla ja
sisätulennolla on useita päätteitä:

\bigskip
Joillakin asemosanoilla on kohdanto-sijamuoto.

\bigskip

\begin{tabular}{@{}|l|l|l|l|} \hline
Yksikön  &         & Monikon      & \\
nimentö  &         & nimentö      & Selitys \\ \hline
valo     &         & valo+t       & \\
joka     &         & jo+tka       & Jo+tka, mi+tkä, ke+tkä. \\ \hline

         &         &              & \\ \hline
Yksikön  & Yksikön & Monikon      & \\
nimentö  & omanto  & omanto       & \\ \hline
nainen   &         & nais+ien     & Monikon omanto monikon vartaloon. \\
valo     &         & valo+jen     & Monikon tunnus on j. \\
takki    &         & takki+en     & Monikon omanto yksikön vartaloon. \\
takka    &         & takka+in     & Monikon omanto yksikön vartaloon. \\
nainen   &         & nais+ten     & \\
valtio   &         & valtio+iden  & \\
valtio   &         & valtio+itten & \\
me       &         & me+idän      & Me+idän, te+idän, he+idän. \\
joka     & jo+nka  &              & Jo+nka, mi+nkä, ke+nkä (ken). \\ \hline

         &          &          & \\ \hline
Yksikön  & Yksikön  & Monikon  & \\
nimentö  & kohdanto & kohdanto & \\ \hline

minä     & minu+t   &          & Minu+t, sinu+t, häne+t, \\
         &          &          & nämä+t (vanhentuntu), kene+t. \\
me       &          & me+idät  & Me+idät, te+idät, he+idät. \\

         &          &          & \\ \hline
Yksikön  & Yksikön  & Monikon  & \\
nimentö  & osanto   & osanto   & \\ \hline
koira    & koira+a  & koir+ia  & \\
valo     &          & valo+ja  & Monikon tunnus on j. \\
puu      & puu+ta   & pu+ita   & \\
hame     & hame+tta &          & Voi myös tulkita hamet+ta. \\ \hline


         &             &             & \\ \hline
Yksikön  & Yksikön     & Monikon     & \\
nimentö  & sisätulento & sisätulento & \\ \hline
koira    & koira+an    & koir+iin    & Yksikössä ääntiö + n. \\
puu      & puu+hun     & pu+ihin     & Yksikössä h + ääntiö + n. \\
hame     & hamee+seen  & hame+isiin  & \\
kaunis   & kaunihi+sen & kaunih+isin & Murteissa. \\ \hline
\end{tabular}
\end{table}
