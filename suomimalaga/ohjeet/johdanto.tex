\section{Johdanto}

Suomi-malaga on malagalla (\verb=http://home.arcor.de/bjoern-beutel/malaga/=)
tehty suomen kielen muoto"-opin kuvaus.

Saksan kielen muoto"-oppi malagaksi: \\
\verb=http://www.linguistik.uni-erlangen.de/~orlorenz/DMM/DMM.html=

Koska tämä on suomen kielen kuvaus, myös kielioppitermit ovat
suomeksi. (-:

Suomi"-malaga on tehty suomen sanojen muuttamiseen perusmuotoon tiedostojen 
indeksointia varten. Sen vuoksi se hyväksyy yleisimpiä kirjoitusvirheitä,
esimerkiksi tyyppiä ''kirjottaa'' ja ''julkasta''.
